\documentclass[a4paper]{article}

%use the english line for english reports
%usepackage[english]{babel}
\usepackage[portuguese]{babel}
\usepackage[utf8]{inputenc}
\usepackage{indentfirst}
\usepackage{graphicx}
\usepackage{verbatim}


\begin{document}

\setlength{\textwidth}{16cm}
\setlength{\textheight}{22cm}

\title{\Huge\textbf{Symple}\linebreak\linebreak\linebreak
\Large\textbf{Relatório Intercalar}\linebreak\linebreak
\includegraphics[height=6cm, width=7cm]{feup.pdf}\linebreak \linebreak
\Large{Mestrado Integrado em Engenharia Informática e Computação} \linebreak \linebreak
\Large{Programação em Lógica}\linebreak
}

\author{\textbf{Grupo 06:}\\ André Duarte - 201100766 \\ João Carlos Santos - 201106760 \\\linebreak\linebreak \\
 \\ Faculdade de Engenharia da Universidade do Porto \\ Rua Roberto Frias, s\/n, 4200-465 Porto, Portugal \linebreak\linebreak\linebreak
\linebreak\linebreak\vspace{1cm}}
%\date{Junho de 2007}
\maketitle
\thispagestyle{empty}

%************************************************************************************************
%************************************************************************************************

\newpage

\section*{Resumo}
Neste trabalho pretende-se implementar um algoritmo para a simulação do jogo de tabuleiro "Symple".


%Descrever muito sumariamente (1-2 parágrafos) o trabalho que está a ser reportado.

%************************************************************************************************
%************************************************************************************************

%*************************************************************************************************
%************************************************************************************************

\section{Introdução}
Descrever os objectivos e motivação do trabalho.

Todas as figuras devem ser referidas no texto. %\ref{fig:codigoFigura}


%Exemplo de código para inserção de figuras
%\begin{figure}[h!]
%\begin{center}
%escolher entre uma das seguintes três linhas:
%\includegraphics[height=20cm,width=15cm]{path relativo da imagem}
%\includegraphics[scale=0.5]{path relativo da imagem}
%\includegraphics{path relativo da imagem}
%\caption{legenda da figura}
%\label{fig:codigoFigura}
%\end{center}
%\end{figure}


\textbf{Info útil}:

Devem ser incluídas referências bibliográficas correctas e completas (consultar os docentes em caso de dúvida). Páginas da wikipedia não são consideradas referências válidas \cite{CodigoSite, CodigoLivro}.

\textit{Para escrever em itálico}

\textbf{Para escrever em negrito}

Para escrever em letra normal

``Para escrever texto entre aspas''

Para fazer parágrafo, deixar uma linha em branco.

Como fazer bullet points:

\begin{itemize}
\item Item1
\item Item2
\end{itemize}

Como enumerar itens:

\begin{enumerate}
\item Item 1
\end{enumerate}

\begin{quote}``Isto é uma citação''\end{quote}

\section{O Symple}
Symple é um jogo de estratégia abstrato em que dois jogadores competem para conseguir ocupar a maior área do tabuleiro com o menor número de grupos possível.

As jogadas são feitas alternadamente entre dois jogadores, de cor Branco e Preto. O jogador Branco é o primeiro a jogar e joga com as peças brancas. O jogador Preto joga com as peças pretas.

No seu turno, um jogador pode escolher entre as seguintes jogadas:
\begin{enumerate}
\item Colocar um pedra no tabuleiro num sítio sem contacto com as outras pedras desta forma criando um novo grupo
\item Fazer crescer todos os grupos com uma pedra. Pedras que toquem em ambos os grupos contam como fazer crescer ambos os grupos. No entanto, se dois grupos crescem em uma pedra e apenas estas se tocam entre os dois grupos, a jogada é legal.
\end{enumerate}

Para equilibrar o jogo existe uma regra extra: se nenhum jogador tiver crescido os seus grupos, o Preto pode crescer todos os seus grupos e criar um novo grupo.

O jogo acaba quando o tabuleiro é preenchido. A pontuação é determinada pelo número de pedras que cada jogador tem no tabuleiro menos 'P' vezes o número de grupos que o jogador tem, em que 'P' é um número par maior ou igual a 4. Com uma penalidade par e um tamanho de tabuleiro ímpar não são possíveis empates.

\section{Representação do Estado do Jogo}
Descrever a forma de representação do estado do tabuleiro (tipicamente uma lista de listas), com exemplificação em Prolog de posições iniciais do jogo, posições intermédias e finais, acompanhadas de imagens ilustrativas.

\section{Visualização do Tabuleiro}
Descrever a forma de visualização do tabuleiro em modo de texto e os predicados Prolog construídos para o efeito. O código (predicado) desenvolvido deve receber como parâmetro a representação do tabuleiro (estado do jogo) e permitir visualizá-lo no ecrã, em modo de texto. Deve ser incluída pelo menos uma imagem correspondente ao output produzido pelo predicado de visualização

\section{Movimentos}
Elencar os movimentos (tipos de jogadas) possíveis e definir os cabeçalhos dos predicados que serão utilizados (ainda não precisam de estar implementados).

\section{Conclusões e Perspectivas de Desenvolvimento}
Que conclui da análise do jogo e da pesquisa bibliográfica realizada? Como vai ser desenvolvido o trabalho? Que parte (\%) do trabalho estima que falta fazer?

\clearpage
\addcontentsline{toc}{section}{Bibliografia}
\renewcommand\refname{Bibliografia}
\bibliographystyle{plain}
\bibliography{myrefs}

\newpage
\appendix
\section{Nome do Anexo}
Código Prolog implementado (representação do estado, cabeçalhos dos predicados para as jogadas e predicado que permite a visualização em modo de texto do tabuleiro).

\end{document}
